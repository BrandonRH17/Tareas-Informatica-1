\documentclass{article}
\usepackage[utf8]{inputenc}
\usepackage{graphicx}

\title{Parcial Recuperacion}
\author{Brandon Rodríguez }
\date{Agosto 2018}

\begin{document}

\maketitle

\section{Pregunta 1}
\\
Su tarea es crear un grafo a partir de estos puentes. Para ello debe:
\begin{itemize}
        \item{Definir el conjunto de nodos}
        \item{Definir el conjunto de vertices}
\end{itemize}
\\
\\
\includegraphics[height=2cm]{1}
\\Los nodos estarán representando a los pedazos de tierra que conectan los puentes. 
\\
\\
\\(A,B,C,D)
\\\
\\Los Vertices consisten en las parejas ordenadas. 
\\
\\
((A,B),(A,C),(C,D),(D,B),(D,A))





\section{Pregunta 2}
\\
 \sum_{i=1}^{n}{i}=\frac{n(n+1)}{2}
\\
\\
Caso Base
\\
\\
 \sum_{i=1}^{n}{i}=\frac{1(1+1)}{2}
 \\
 \\
  \sum_{i=1}^{n}{i}=1
 \\
 \\
 1=1
 \\
 \\
Hipótesis Inductiva
\\
\\
 \sum_{i=1}^{n}{i}=\frac{n(n+1)}{2}
\\
\\
Demostración
\\
\\
 \sum_{i=1}^{n+1}{i}=\frac{(n+1)(n+2)}{2}
 \\
 \\
  \sum_{i=1}^{n}{i}=1+2+3+4..+n+(n+1)
 \\
 \\
  =\frac{n(n+1)}{2}+\frac {n+1}{1}
 \\
 \\
 =\frac{n(n+1)+2(n+1)}{2}
 \\
 \\
  =\frac{(n+1)(n+2)}{2}=\frac{(n+1)(n+2)}{2}
 \\
 \\
 \section{Pregunta 3}
 \\
 Definir inductivamente la funcion $\sum(n)$ para numeros naturales unarios la cual tiene
el efecto de calcular la suma de $1$ hasta $n$. En otras palabras:
\[
        \sum(n)=1+2+3+4+\ \ldots\ +n
\]
Puede apoyarse de la suma $\oplus$ de numeros naturales unarios para su definici\'on:
\[
        a\oplus b =
                \left\{
                        \begin{array}{ll}
                                b  & \mbox{si } a = 0 \\
                                s(i\oplus b) & \mbox{si } a = s(i)
                        \end{array}
                \right.
\]
\\
\\
\section{Pregunta 4}
\\
Demostrar por medio de inducci\'on la conmutatividad de la suma de
numeros naturales unarios: $a\oplus b = b\oplus a$
\\
\\
Caso Base
\\
\\
0 + n = n + 0
\\
\\
n = n
\\
\\
Hipótesis Inductiva
\\
\\
\sum(n)=1+2+3+4+\ \ldots\ +n
\\
\\
Demostración
\\
\\
n + s(m) = s(n + m) 
\\
\\
= s(m + n)
\\
\\
\\
\\
\\
\\
\section{Pregunta 5}
\\
\\
Dada la funci\'on $a\geq b$ para numeros naturales unarios:
\[
        a\geq b =
                \left\{
                        \begin{array}{ll}
                                s(o)  & \mbox{si } b = o \\
                                o & \mbox{si } a = o \\
                                i\geq j & \mbox{si } a = s(i)\ \&\ b = s(j)
                        \end{array}
                \right.
\]
Demostrar utilizando inducci\'on que $((n\oplus n)\geq n) = s(o)$. Puede
hacer uso de la asociatividad y comutabilidad de la suma de numeros
unarios para su demostraci\'on.
\\
\\



\end{document}
