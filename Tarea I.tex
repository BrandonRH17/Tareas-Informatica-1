\documentclass[14pt]{article}
\usepackage[spanish]{babel}
\author{Brandon Rodriguez}
\title{Tarea 1}

\begin{document}

\maketitle
\section{Ejercicio 2}
1. El conjunto de nodos del grafo es: 
{1,2,3,4,5,6}
\\

2. El conjunto de vertices del grafo es: 
(1,2),(1,3),(1,4),(1,5),(1,6)
(2,1),(2,3),(2,4),(2,5),(2,6)
(3,1),(3,2),(3,4),(3,5),(3,6)
(4,1),(4,2),(4,3),(4,5),(4,6)
(5,1),(5,2),(5,3),(5,4),(5,6)
(6,1),(6,2),(6,3),(6,4),(6,5)

\section{Ejercicio 3}
1. Camino tipo grafo
\\
2. Busqueda de Uniones
\\
3. Los números al ser colocados de esa manera, se asegura que despúes de una rotación de 90
grados a cualquier dirección, caerá un número distinto.

\end{document}
\
\