\documentclass[14pt]{article}
\usepackage[spanish]{babel}
\author{Brandon Rodriguez}
\title{Tarea 3}

\begin{document}

\maketitle
\section{Ejercicio 1}
$ssso + sssso \newline
s(ssso + ssso) \newline
s(s(ssso + sso)) \newline
s(s(s(ssso + so))) \newline
s(s(s(s(ssso+o)))) \newline
s(s(s(s(s(sso+o))))) \newline
s(s(s(s(s(s(so+o)))))) \newline 
s(s(s(s(s(s(s(o))))))) \newline$
7

\section{Ejercicio 2}
Caso Base
\\
Por axiomas pasados, se sabe que cualquier número multiplicado por cero es igual a cero.
\\
\\
Caso Inductivo\\
$a\otimes b = a + (a*(b-1)) \newline
a\otimes b = a + (ab-a) \newline
a\otimes b = a + ab-a \newline
a\otimes b = ab \newline
a\otimes b = a\otimes b $
\\

\section{Ejercicio 3}
$s(s(s(0))) \otimes 0$ \newline
 \newline
Axioma = n por cero es 0
\\
\\

{$s(s(s(0)))\otimes s(0)$} \newline
{$s(s(s(0)))\otimes s(0)$} = ssso + (sss0 *(s0-s0) \newline
{$s(s(s(0)))\otimes s(0)$} = ssso + (sss0* (0) \newline
{$s(s(s(0)))\otimes s(0)$} = ssso \newline
\\
\\
\\

{$s(s(s(0)))\otimes s(s(0))$} \newline
{$s(s(s(0)))\otimes s(s(0))$} = sss0 + (sss0 * (ss0-s0) \newline
{$s(s(s(0)))\otimes s(s(0))$} =sss0 + (sss0*s0) \newline
{$s(s(s(0)))\otimes s(s(0))$} = sss0 + sss0 \newline
{$s(s(s(0)))\otimes s(s(0))$} = ssssss0 \newline
\\
\\
\section{Ejercicio 4}


$a\oplus s(s(0))=s(s(a))$ \newline
$a\oplus s(s(0))=s(0) +s(a)$ \newline
$a\oplus s(s(0))=s(s(0)) +a$ \newline
$a\oplus s(s(0))=a\oplus s(s(0))$ \newline
\newline

$a \otimes b = b \otimes a$ \newline
$sa \otimes sb = sb \otimes sa$ \newline
$s(a \otimes b) = s(b \otimes a)$ \newline
$s(a \otimes b)/s = s(b \otimes a)/s$ \newline
$a \otimes b = b \otimes a$ \newline
\newline


Caso base C=0 \newline
$a \otimes (b \otimes 0)=(a\otimes b)\otimes 0$ \newline
$a \otimes 0=(a\otimes b)\otimes 0$ \newline
Axioma = n por cero es 0
\\
$0=0$ \newline
\newline
Caso Inductivo \newline
$a \otimes (b \otimes c)=(a\otimes b)\otimes c$ \newline
$a \otimes bc=ab\otimes c$ \newline
$abc=abc$ \newline
\newline





\end{document}
\
\